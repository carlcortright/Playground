\documentclass[11pt,english]{article}
\usepackage[T1]{fontenc}
\usepackage{babel}
\usepackage{textcomp}

\begin{document}

% Define document title and author
\title{Modeling Object Behavior Near Black Holes}
\author{
  Carl Cortright\\
  \texttt{SID: 104731007 Recitation Number: 212 Section Number: 130}
  \and
  Samuel Coyle\\
  \texttt{SID: 105477414 Recitation Number: 243 Section Number: 140}
  \and
  Nelson Botsford\\
  \texttt{SID: 103627718 Recitation Number: 243 Section Number 140}
}
\maketitle

\section*{Introduction}

The study of black holes has classically been restricted to the field of theoretical physics, but lately with movies like Interstellar and Gravity the science has begun to capture the public imagination. While it is impossible to model the inner workings of a black hole beyond the event horizon, we can use differential equations to model the paths of object moving near a black hole, including their relativistic properties. In this report, we analyze the path of an astronaut (Matthew McConaughey) as he descends towards the event horizon from our perspective on a ship whose mission is to seed humanity\textquotesingle s new home planet. Before we lost contact, we were able to model the path of the astronaut with the following differential equation.

\begin{center}
$\frac{dx}{dt} = (\frac{1}{x(t)} - 1) \frac{1}{\sqrt{x(t)}}$
\end{center}

\section*{Understanding the Astronaut\textquotesingle s Path}

To better understand the path of the astronaut, we must first classify the differential equation as autonomous, non-linear, homogeneous, and separable. The astronaut\textquotesingle s speed is the derivative of his position, and the equation is autonomous/homogeneous, meaning that his speed is entirely dependent on his proximity from the center of the black hole.

As mentioned earlier, the speed of the astronaut is the time derivative of position, meaning the differential equation given is the speed of the astronaut as he approaches the black hole. If we have the position of the astronaut, then we are also able to measure his velocity, making it trivial to find his initial velocity.

\begin{center}
$x(0) = 2$,  $v_o = (\frac{1}{2} - 1)\frac{1}{\sqrt{2}} = \frac{-1}{2\sqrt{2}} = -\sqrt{2}$
\end{center}

\section*{Analyzing Existence and Uniqueness of Solution Curves}

Before we decide to use our model as a model for the astronaut\textquotesingle s path, we must first verify that it produces sufficiently unique solutions in proximity of where we want to do our analysis. A useful theorem for defining existence and uniqueness of solutions is Picard\textquotesingle s Theorem which states that the derivative is continuous on a specific rectangle and the second partial derivative with respect to the dependent variable is continuous on that same rectangle, then there exists a unique solution within some interval on the rectangle. Using Picard\textquotesingle s, we first need to find the second partial with respect to x:

\begin{center}
  $\frac{\partial f}{\partial x} = \frac{d}{dx} (\frac{1}{x(t)} - 1) \frac{1}{\sqrt{x(t)}}$
\end{center}

Using this derivative, we are able to tell that there will be a unique solution on any rectangle where all values of x are greater than 0. 

\section*{Conclusion}

Our astronaut will never reach the event horizon of the black hole, at least from the perspective of observers on the space station. Instead he will approach it, quickly at first, but he will soon slow down and be doomed to spend an eternity gravitating towards the event horizon. 

% Your document ends here!
\end{document}
